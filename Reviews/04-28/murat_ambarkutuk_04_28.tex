\documentclass[11pt]{article}
\usepackage[margin=1in]{geometry}
\renewcommand{\baselinestretch}{1}
\begin{document}
\thispagestyle{empty}
%%% Summary
\begin{flushright}
	\small{Murat Ambarkutuk \{murata@vt.edu\}, 04/28/2016}
\end{flushright}
\section*{Rapid Natural Scene Categorization in the Near Absense of Attention}
In their paper the authors investigate a phenomena regarding human cognitive skills.
The phenomena implies that human cognitive skills are limited to the extent of attention for achieving high level tasks.
However, the findings of the experimentation provided in the paper implies the contrary.
% \section*{Experimentation}
The experimentatal setup provided in the paper consists of three conditions: Main experiment and two control experiments.
a
The main experiment is a cognition task where visual attention is required to carry out where the task is to distinguish if the combination of five letters (letter L, and letter F) shown in the screen are the same or not.
The letters then masked with letter F after a short period of time; and the subjects are asked to type S or D (representing the words same and different).

The first control experiment, on the other hand, is a peripheral task where the role of attention is assessed, while the last experiment combines the former two.
The first control experiment, referred as the Peripheral Task in the paper, consists of three different sub-tasks: Natural Scene categorization, Peripheral Letter Discrimination, and Peripheral Color Pattern Discrimination.
The subjects were trained to hold down the mouse button as the task continues and to release it once they detect a target.

The dual task is meant to be carry out the former two experiments concurrently to scrutinize the central attention and peripheral attention.
The goal of the dual task experimentation to evaluate the role of attention by comparing the results of carrying out the categorization task only and carrying out the categorization task while accomplishing another task concurrently.
% The left hand utilized for the central attention (typing the letter S and D), while the right hand u
\section*{Results and Conclusions}
%%% Results
%% Explaint experimental setup
%% Underline the results
The paper provides a compherensive set of results regarding the experiments conducted.
The subjects performed about 77\% varying from 68\% and 82\% in the central attention tasks while if the peripheral cognitive task is carried out only, the subjects averaged 76\%.
As for the dual task experimentation, the first metric measured was the overall response time of the user for the central attention task.
The authors noted that the overall response time didn't change while the subjects were carrying out the central attention test with and without the peripheral test, which implies that the subjects' attention was completely employed for the central attention task.
Another important implication could be drawn from the dual task experiment was the fact that test subjects were able to carry out the peripheral test with a similar performance if the subjects carried out the peripheral test alone.
This implies that higher level cognitive tasks can be rapidly accomplished under the absense of visual attention.

On the contrary of the common belief that human perception system is limited with the level of attention in the higher level cognitive tasks, the study shows that the human perceptive and cognitive systems are able to categorize two different visual stimuli.
However, when it comes to being able to handle more than one task at time same time or switching tasks quickly, there are some results from other research studies implying that performance of multitasking heavily depends on many parameters, such as the levels of melatonin hormone in the body; thus, training would not improve the performance in short period of time.
For that reason, it would be better if there were more test subjects participated in the study to eliminate such tendencies in the results.
% \bibliographystyle{unsrt}
% \bibliography{bibliography}
\end{document}
