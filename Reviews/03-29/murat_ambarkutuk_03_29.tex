\documentclass[11pt]{article}
\usepackage[margin=1in]{geometry}
\renewcommand{\baselinestretch}{0.95}
\begin{document}
\thispagestyle{empty}
%%% Summary
\begin{flushright}
	\small{Murat Ambarkutuk \{murata@vt.edu\}, 03/24/2016}
\end{flushright}
\section*{Learning To Detect Unseen Object Classes by Between-Class Attribute Transfer}
The problem that the authors are tackling is that of detection in a framework in which the training data and test data are disjoint.
In other words, the paper formulates an object detection method which can handle unseen object classes.
% In other words, the paper formulates a method which enables learning algorithms to have disjoint training and test set.

% The authors denote that the conventional representation of datasets impedes learning algorithms from discovering new categories during test time.
The authors note that it is not feasible to solve such problems with the conventional learning paradigm.
Thus, it is required to have a new approach to abstract the data (possibly, semantically higher level than the conventional representation).
The authors suggest the use of attributes to handle disjoint training sets and test sets.
The major difference of using the attributes is to decouple these sets.
% The paper proposes using attributes as a higher level representation for the images, which is shared between classes.
The attributes may be the shape, color or even the geographical information of the object, which are semantically more meaningful than raw features extracted from the images.
Furthermore, it is crucial to note that the proposed method enforces that the attributes are shared among classes.
One example of different attributes provided in the paper is the descriptions of the bee, tiger and zebra classes.
While three example categories share the attribute stripe, the color information captures the differences among the classes.
In order to use attributes in classification tasks, the paper proposes two different approaches using attributes as the major component in their classification scheme.

The first approach is Direct Attribute Prediction (DAP) which uses attributes as a decoupling layer between the images and the labels.
During training, each output class layer invokes a labeling in the attribute layer.
The authors note that any supervised learning scheme can be employed with the proposed method.
Moreover, the attribute values are used to infer the class label at test time.

Indirect Attribute Prediction (IAP), the second proposed approach, essentially employs a similar process.
The main difference of the IAP approach is that the attributes layer connects classes which are known at training time and not.
The learning scheme is the conventional multi-class classification.

\section*{Results and Conclusions}
%%% Results
%% Explaint experimental setup
%% Underline the results
The IAP and DAP approaches are evaluated seperately with a novel dataset tailored for the task.
A non-linear SVM and multi-class logistic regression was trained for DAP and IAP, respectively.

One measure used for evaluation purposes was the mean of the diagonal of the confusion matrix.
40.5\% and 27.8\% multi-clas accuracy scores were obtained with the mentioned measure for DAP and IAP, respectively.
% The confussion matrix in the figure-4 depicts that DAP is more accurate, while IAP is more precise.
This measure implies that DAP is more accurate than IAP in multi-class classification tasks.
In other words, in overall performance comparison, DAP successfully identifies more classes.
However, IAP shows significantly higher confidence scores at classes it infers correctly.
One extension that could be made for the paper would be using both approaches simulatenously and fusing the outputs of the approaches.
By doing this, the expected result of this extension would be an increase in both accuracy and precision of the proposed method.

To compare the proposed method with one of the avaiable methods, the authors implemented a one-shot learning scheme along with a nearest-neighbor classification method and a multi-class classification method.
% The one-shot algorithm performed relatively weaker than the proposed method if the training
The one-shot learning algorithm showed accuracy of 14.3\% and 18.9\% with 1 and 10 training images.
The authors commented that these results prove how hard the problem of interest is.
Furthermore, the multi-class classification method performed with an accuracy of 65\%; however it requires more data than the proposed method to achieve this level of accuracy.
As a conclusion, the attribute-based learning scheme has the potential to perform in cases where no or lack training data available.
% \bibliographystyle{unsrt}
% \bibliography{bibliography}
\end{document}
